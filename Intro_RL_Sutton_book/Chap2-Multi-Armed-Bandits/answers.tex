\documentclass[10pt,a4paper]{article}
\usepackage[utf8]{inputenc}
\usepackage{amsmath}
\usepackage{amsfonts}
\usepackage{amssymb}
\usepackage{hyperref}

\hypersetup{
    colorlinks=true,
    linkcolor=blue,
    filecolor=magenta,      
    urlcolor=cyan,
    pdftitle={Overleaf Example},
    pdfpagemode=FullScreen,
    }

\title{Chapter 2 - Multi-armed Bandits Answers}
\author{Stéphane Liem NGUYEN}
\begin{document}
\maketitle

Exercises with (\textbf{\textit{corrected}}) were corrected based on the \href{http://incompleteideas.net/book/errata.html}{Errata}.

These are my own answers and mistakes or errors are possible.

\paragraph{\textit{Exercise 2.1} (p. 28)} In $\epsilon$-greedy action selection, for the case of two actions and $\epsilon = 0.5$, what is
the probability that the greedy action is selected?

\bigskip
Recall that at each time step, $\epsilon$-greedy action selection methods select with probability $\epsilon$ a random action uniformly out of $k$ actions (if there's $k$ actions) and with probability $1-\epsilon$ one of the greedy actions with ties broken arbitrarily (for instance randomly).

From the recall, we can say that the probability of the greedy action to be selected   is $1-\epsilon = 0.5$.

\paragraph{\textit{Exercise 2.2: Bandit example} (p. 30)} Consider a $k$-armed bandit problem with $k = 4$ actions,
denoted $1$, $2$, $3$, and $4$. Consider applying to this problem a bandit algorithm using
$\epsilon$-greedy action selection, sample-average action-value estimates, and initial estimates
of $Q_1(a) = 0$, for all a. Suppose the initial sequence of actions and rewards is $A_1 = 1,
R_1 = -1, A_2 = 2, R_2 = 1, A_3 = 2, R_3 = -2, A_4 = 2, R_4 = 2, A_5 = 3, R_5 = 0$. On some of these time steps the $\epsilon$ case may have occurred, causing an action to be selected at
random. On which time steps did this definitely occur? On which time steps could this
possibly have occurred?

\bigskip
We already recalled what were the $\epsilon$-greedy action selection methods. We can recall quickly that sample-average action-value estimates compute estimates $Q_n(a)$ based on an empirical mean of the rewards received when the action $a$ is taken. The empirical means can be computed incrementally as explained in the book.

To know in what time step did the $\epsilon$ case definitely occur, we use the information that the decision-maker only picks a non-greedy action (not actions with highest current estimates) when exploring.

For this reason, we also compute estimates after each action-selection:
\begin{enumerate}
\item Take action $1$, receive reward $-1$ and update estimates with $Q_2(1)=-1$ and keep $Q_1(a)=0,\, \forall a \neq 1$. For the next action selection, the action $1$ cannot be selected from greedy selection because its estimate is lower than the rest. The action $1$ can only be picked from exploration for the next action selection.

\item Take action $2$, receive reward $1$ and update estimates with $Q_2(2)=1$ and keep $Q_2(1)=-1,\, Q_1(a)=0,\, \forall a \notin \{1, 2\}$. For the next time step, if we pick another action than $2$ then it would be due to exploration.

\item Take action $2$, receive reward $-2$ and update estimates with $Q_3(2)=\frac{1 + (-2)}{2} = -0.5$ and keep $Q_2(1)=-1,\, Q_1(a)=0,\, \forall a \notin \{1, 2\}$. For the next action selection, the action $2$ can only be picked from exploration because its estimate is lower than the estimates of actions $3$ and $4$.

\item Take action $2$, receive reward $2$ and update estimates with $Q_4(2)=\frac{1 + (-2) + 2}{3} = \frac{1}{3}$ and keep $Q_2(1)=-1,\, Q_1(a)=0,\, \forall a \notin \{1, 2\}$. For the next time step, if we pick another action than $2$ then it would be due to exploration. 

\item Take action $3$, receive reward $0$ and update estimates with $Q_2(3)=0$ and keep $Q_4(2) = \frac{1}{3},\, Q_2(1)=-1,\, Q_1(4)=0$.

\end{enumerate}

The $\epsilon$ case could have occurred at all time steps but it definitely occurred  for time step $4$ and time step $5$.

\paragraph{\textit{Exercise 2.3} (p. 30)} In the comparison shown in Figure $2.2$, which method will perform best in
the long run in terms of cumulative reward and probability of selecting the best action?
How much better will it be? Express your answer quantitatively.

\bigskip
Action-value method with $\epsilon=0.01$ will perform best in the long run in terms of cumulative reward and probability of selecting the best action.

The probability of selecting the best action $a_*$ for $\epsilon$-greedy methods with $\epsilon > 0$ (with sample-averages and on stationary problems) can be written as the probability of taking an optimal action when exploring ($\epsilon$ case, action selected at random) plus the probability of taking a greedy action (we can add them because in one single action selection, we cannot behave greedily at the same time as non-greedily. Moreover the second part with greedy action assumes that the estimates of the optimal actions converge to the action values).

\begin{equation}
\lim_{t\to\infty}\mathbb{P}[A_t = a_*] = \epsilon\cdot\frac{1}{k}+ 1 - \epsilon
\end{equation}

This formula does not work for $\epsilon = 0$ (greedy) because if $\epsilon=0$, some actions can be selected a finite number of times asymptotically. As consequence, some estimates might not converge to the true action values.

For $10$-armed testbed, $k=10$ and by applying the formula with $\epsilon=0.1$ we get $0.01 + 1 - 0.1 = 0.91$. If we apply the formula with $\epsilon=0.01$ then we get $0.001 + 1 - 0.01 = 0.991$.


\paragraph{\textit{Exercise 2.4} (p. 33)} If the step-size parameters, $\alpha_n$, are not constant, then the estimate $Q_n$ is
a weighted average of previously received rewards with a weighting different from that
given by ($2.6$). What is the weighting on each prior reward for the general case, analogous
to ($2.6$), in terms of the sequence of step-size parameters?

\bigskip
Recall that in this section, the book focused on estimating/tracking incrementally the value of a single action for nonstationary problem and this was done by using a constant-step-size parameter in the incremental update rule that we rewrite below
\begin{equation}
Q_{n+1} \doteq Q_n + \alpha_n [R_n - Q_n]
\end{equation}
where $R_n$ is the reward received after the $n$-th selection of the action and $\alpha_n=\frac{1}{n}$ for the incremental implementation of sample-averages (empirical mean) or $\alpha_n=\alpha \in (0, 1]$ for the exponential recency-weighted average.

Equation $(2.6)$ showed that if we update estimates using exponential recency-weighted average then $Q_{n+1}$ was a weighted average of past rewards and the initial estimate $Q_1$.

We can try to do the same but for the general case where $\alpha_n$ is not necessarily $\frac{1}{n}$ or a constant $\alpha \in (0,1]$.
\begin{equation}
\begin{split}
Q_{n+1} &= Q_n + \alpha_n [R_n - Q_n] = \alpha_n R_n + (1-\alpha_n) Q_n\\
&= \alpha_n R_n + (1-\alpha_n) [\alpha_{n-1} R_{n-1} + (1-\alpha_{n-1})Q_{n-1}]\\
&= \alpha_n R_n + (1-\alpha_n) \alpha_{n-1} R_{n-1} + (1-\alpha_n)\cdot (1-\alpha_{n-1})Q_{n-1}\\
&= Q_1 \prod_{k=1}^n (1-\alpha_k) + \sum_{i=1}^n \alpha_i R_i \prod_{j=i+1}^n (1-\alpha_j)
\end{split}
\end{equation}

Each reward $R_i$ is weighted by $\alpha_i \prod_{j=i+1}^n (1-\alpha_j)$.

\bigskip
We can verify by recurrence that it's a weighted sum, i.e $\prod_{k=1}^n (1-\alpha_k) + \sum_{i=1}^n \alpha_i \prod_{j=i+1}^n (1-\alpha_j) = 1$. For $n=1$ we have
\begin{equation}
1-\alpha_1 + \alpha_1 = 1
\end{equation}
Suppose the formula is true for $n$, let's prove that it holds for $n+1$. Let $H_n = \prod_{k=1}^n (1-\alpha_k) + \sum_{i=1}^n \alpha_i \prod_{j=i+1}^n (1-\alpha_j) = 1$

\begin{equation}
\prod_{k=1}^{n+1} (1-\alpha_k) + \sum_{i=1}^{n+1} \alpha_i \prod_{j=i+1}^{n+1} (1-\alpha_j) = H_n \cdot (1-\alpha_{n+1}) + \alpha_{n+1} = 1
\end{equation}

From the general formula, we can obtain equation $(2.6)$ of the book for exponential recency-weighted average as well as the sample-average methods where $R_i$'s are weighted by $\frac{1}{i} \cdot \prod_{j=i+1}^n \left(1-\frac{1}{j}\right) = \frac{1}{i} \cdot \frac{n-1}{n} \cdot \frac{n-2}{n-1} \cdot \hdots \cdot \frac{i}{i+1} = \frac{1}{n}$

\paragraph{\textit{Exercise 2.5 (programming)} (p. 33)} Design and conduct an experiment to demonstrate the
difficulties that sample-average methods have for nonstationary problems. Use a modified version of the $10$-armed testbed in which all the $q_*(a)$ start out equal and then take independent random walks (say by adding a normally distributed increment with mean $0$ and standard deviation $0.01$ to all the $q_*(a)$ on each step). Prepare plots like Figure $2.2$
for an action-value method using sample averages, incrementally computed, and another action-value method using a constant step-size parameter, $\alpha = 0.1$. Use $\epsilon = 0.1$ and
longer runs, say of 10,000 steps.

\paragraph{\textit{Exercise 2.6: Mysterious Spikes} (p. 35)} The results shown in Figure $2.3$ should be quite reliable because they are averages over $2000$ individual, randomly chosen $10$-armed bandit tasks.
Why, then, are there oscillations and spikes in the early part of the curve for the optimistic method? In other words, what might make this method perform particularly better or worse, on average, on particular early steps?

\paragraph{\textit{Exercise 2.7: Unbiased Constant-Step-Size Trick} (\textit{corrected}) (p. 35)} In most of this chapter we have used
sample averages to estimate action values because sample averages do not produce the
initial bias that constant step sizes do (see the analysis leading to ($2.6$)). However, sample averages are not a completely satisfactory solution because they may perform poorly on nonstationary problems. Is it possible to avoid the bias of constant step sizes while retaining their advantages on nonstationary problems? One way is to use a step size of
\begin{equation}
\beta_n \doteq \alpha/\bar{o}_n
\end{equation}

to process the $n$th reward for a particular action, where $\alpha > 0$ is a conventional constant step size, and $\bar{o}_n$ is a trace of one that starts at $0$:
\begin{equation}
\bar{o}_n \doteq \bar{o}_{n-1} + \alpha \cdot (1 - \bar{o}_{n-1}),\quad \textrm{for } n > 0,\quad \textrm{with } \bar{o}_0 \doteq 0
\end{equation}

Carry out an analysis like that in ($2.6$) to show that $Q_n$ is an exponential recency-weighted average \textit{without initial bias}.

\paragraph{\textit{Exercise 2.8: UCB Spikes} (p. 36)} In Figure $2.4$ the UCB algorithm shows a distinct spike
in performance on the $11$th step. Why is this? Note that for your answer to be fully satisfactory it must explain both why the reward increases on the $11$th step and why it decreases on the subsequent steps. Hint: If $c = 1$, then the spike is less prominent.

\paragraph{\textit{Exercise 2.9} (p. 37)} Show that in the case of two actions, the soft-max distribution is the same
as that given by the logistic, or sigmoid, function often used in statistics and artificial
neural networks.

\paragraph{\textit{Exercise 2.10} (p. 41)} Suppose you face a $2$-armed bandit task whose true action values change randomly from time step to time step. Specifically, suppose that, for any time step, the true values of actions $1$ and $2$ are respectively $10$ and $20$ with probability $0.5$ (case
A), and $90$ and $80$ with probability $0.5$ (case B). If you are not able to tell which case you face at any step, what is the best expected reward you can achieve and how should
you behave to achieve it? Now suppose that on each step you are told whether you are
facing case A or case B (although you still don't know the true action values). This is an associative search task. What is the best expected reward you can achieve in this task, and how should you behave to achieve it?

\paragraph{\textit{Exercise 2.11 (programming)} (p. 44)} Make a figure analogous to Figure $2.6$ (parameter study) for the nonstationary case outlined in \textbf{\textit{Exercise 2.5}}. Include the constant-step-size $\epsilon$-greedy algorithm with
$\alpha=0.1$. Use runs of 200,000 steps and, as a performance measure for each algorithm and parameter setting, use the average reward over the last 100,000 steps.
\end{document}