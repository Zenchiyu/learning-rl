\documentclass[10pt,a4paper]{article}
\usepackage[utf8]{inputenc}
\usepackage{amsmath}
\usepackage{amsfonts}
\usepackage{amssymb}
\usepackage{hyperref}

\hypersetup{
    colorlinks=true,
    linkcolor=blue,
    filecolor=magenta,      
    urlcolor=cyan,
    pdftitle={Overleaf Example},
    pdfpagemode=FullScreen,
    }

\title{Chapter 2 - Multi-armed Bandits Exercises}
\author{Stéphane Liem NGUYEN}
\begin{document}
\maketitle

Exercises with (\textbf{\textit{corrected}}) were corrected based on the \href{http://incompleteideas.net/book/errata.html}{Errata}

\paragraph{\textit{Exercise 2.1} (p. 28)} In $\epsilon$-greedy action selection, for the case of two actions and $\epsilon = 0.5$, what is
the probability that the greedy action is selected?

\paragraph{\textit{Exercise 2.2: Bandit example} (p. 30)} Consider a $k$-armed bandit problem with $k = 4$ actions,
denoted $1$, $2$, $3$, and $4$. Consider applying to this problem a bandit algorithm using
"$\epsilon$-greedy action selection, sample-average action-value estimates, and initial estimates
of $Q_1(a) = 0$, for all a. Suppose the initial sequence of actions and rewards is $A_1 = 1,
R_1 = -1, A_2 = 2, R_2 = 1, A_3 = 2, R_3 = -2, A_4 = 2, R_4 = 2, A_5 = 3, R_5 = 0$. On some of these time steps the $\epsilon$ case may have occurred, causing an action to be selected at
random. On which time steps did this definitely occur? On which time steps could this
possibly have occurred?

\paragraph{\textit{Exercise 2.3} (p. 30)} In the comparison shown in Figure $2.2$, which method will perform best in
the long run in terms of cumulative reward and probability of selecting the best action?
How much better will it be? Express your answer quantitatively.

\paragraph{\textit{Exercise 2.4} (p. 33)} If the step-size parameters, $\alpha_n$, are not constant, then the estimate $Q_n$ is
a weighted average of previously received rewards with a weighting different from that
given by ($2.6$). What is the weighting on each prior reward for the general case, analogous
to ($2.6$), in terms of the sequence of step-size parameters?

\paragraph{\textit{Exercise 2.5 (programming)} (p. 33)} Design and conduct an experiment to demonstrate the
difficulties that sample-average methods have for nonstationary problems. Use a modified version of the $10$-armed testbed in which all the $q_*(a)$ start out equal and then take independent random walks (say by adding a normally distributed increment with mean $0$ and standard deviation $0.01$ to all the $q_*(a)$ on each step). Prepare plots like Figure $2.2$
for an action-value method using sample averages, incrementally computed, and another action-value method using a constant step-size parameter, $\alpha = 0.1$. Use $\epsilon = 0.1$ and
longer runs, say of 10,000 steps.

\paragraph{\textit{Exercise 2.6: Mysterious Spikes} (p. 35)} The results shown in Figure $2.3$ should be quite reliable because they are averages over $2000$ individual, randomly chosen $10$-armed bandit tasks.
Why, then, are there oscillations and spikes in the early part of the curve for the optimistic method? In other words, what might make this method perform particularly better or worse, on average, on particular early steps?

\paragraph{\textit{Exercise 2.7: Unbiased Constant-Step-Size Trick} (\textit{corrected}) (p. 35)} In most of this chapter we have used
sample averages to estimate action values because sample averages do not produce the
initial bias that constant step sizes do (see the analysis leading to ($2.6$)). However, sample averages are not a completely satisfactory solution because they may perform poorly on nonstationary problems. Is it possible to avoid the bias of constant step sizes while retaining their advantages on nonstationary problems? One way is to use a step size of
\begin{equation}
\beta_n \doteq \alpha/\bar{o}_n
\end{equation}

to process the $n$th reward for a particular action, where $\alpha > 0$ is a conventional constant step size, and $\bar{o}_n$ is a trace of one that starts at $0$:
\begin{equation}
\bar{o}_n \doteq \bar{o}_{n-1} + \alpha \cdot (1 - \bar{o}_{n-1}),\quad \textrm{for } n > 0,\quad \textrm{with } \bar{o}_0 \doteq 0
\end{equation}

Carry out an analysis like that in ($2.6$) to show that $Q_n$ is an exponential recency-weighted average \textit{without initial bias}.

\paragraph{\textit{Exercise 2.8: UCB Spikes} (p. 36)} In Figure $2.4$ the UCB algorithm shows a distinct spike
in performance on the $11$th step. Why is this? Note that for your answer to be fully satisfactory it must explain both why the reward increases on the $11$th step and why it decreases on the subsequent steps. Hint: If $c = 1$, then the spike is less prominent.

\paragraph{\textit{Exercise 2.9} (p. 37)} Show that in the case of two actions, the soft-max distribution is the same
as that given by the logistic, or sigmoid, function often used in statistics and artificial
neural networks.

\paragraph{\textit{Exercise 2.10} (p. 41)} Suppose you face a $2$-armed bandit task whose true action values change randomly from time step to time step. Specifically, suppose that, for any time step, the true values of actions $1$ and $2$ are respectively $10$ and $20$ with probability $0.5$ (case
A), and $90$ and $80$ with probability $0.5$ (case B). If you are not able to tell which case you face at any step, what is the best expected reward you can achieve and how should
you behave to achieve it? Now suppose that on each step you are told whether you are
facing case A or case B (although you still don't know the true action values). This is an associative search task. What is the best expected reward you can achieve in this task, and how should you behave to achieve it?

\paragraph{\textit{Exercise 2.11 (programming)} (p. 44)} Make a figure analogous to Figure $2.6$ (parameter study) for the nonstationary case outlined in \textbf{\textit{Exercise 2.5}}. Include the constant-step-size $\epsilon$-greedy algorithm with
$\alpha=0.1$. Use runs of 200,000 steps and, as a performance measure for each algorithm and parameter setting, use the average reward over the last 100,000 steps.
\end{document}